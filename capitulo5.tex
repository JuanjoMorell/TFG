%!TEX root = proyecto.tex

\chapter{Conclusiones y vías futuras}

Durante el desarrollo de este trabajo se han estudiado y probado la utilización de cuatro técnicas sobre la tarea de detección facial con mascarilla, para asegurar el cumplimiento de las normas COVID-19 impuestas por la OMS. Estas técnicas son las siguientes: Haa-like feature, Facial Landmarks, Mediapipe y Transfer Learning.

Según los resultados obtenidos durante el estudio se concluye en que las técnicas presentadas consiguen realizar detecciones de rostros con mascarillas, cumpliendo así el primer objetivo impuesto. Los mejores resultados se logran con el primer prototipo, centrado en el uso de \textit{Haar-like features} combinado con \textit{Machine Learning} para la creación de un modelo apto para medir el cumplimiento de las normas COVID-19, durante una ejecución en tiempo real. De igual forma, el resto de técnicas son capaces de realizar dicha tarea, destacando el buen rendimiento del prototipo con Mediapipe y los malos resultados de \textit{Facial Landmark}, incapaz de detectar el rostro con mascarilla en muchas ocasiones.

[Hablar de todos los objetivos y prototipos]

A lo largo del desarrollo de este trabajo se ha aprendido a emplear herramientas como Python, Tensorflow, frameworks de Python (Mediapipe, Dlib, Numpy) y OpenCV. Además, de profundizar en términos e ideas del \textit{Machine Learning}, visión artificial y \textit{Deep Learning}.

A modo de cierre, se pueden desatacar las siguientes vías futuras para continuar este trabajo:
\begin{itemize}
	\item Probar en más dispositivos
	\item Ampliar los prototipos, creando modelos más completos
	\item Utilizar servicios en la nube para mejorar su uso.
	\item Realizar el estudio de más técnicas.
	\item Añadir más normas COVID-19 en los prototipos.
\end{itemize}