%!TEX root = proyecto.tex

\chapter{Conclusiones y vías futuras}

Durante el desarrollo de este trabajo se han estudiado y probado la utilización de cuatro técnicas sobre la tarea de detección facial con mascarilla, para asegurar el cumplimiento de las normas COVID-19 impuestas por la OMS (Organización Mundial de la Salud). Estas técnicas son las siguientes: Haa-like feature, Facial Landmarks, Mediapipe y Transfer Learning.

Según los resultados obtenidos durante el estudio se concluye en que las técnicas presentadas consiguen realizar detecciones de rostros con mascarillas, cumpliendo así el primer objetivo impuesto. Los mejores resultados se logran con el primer prototipo, centrado en el uso de \textit{Haar-like features} combinado con \textit{Machine Learning} para la creación de un modelo apto para medir el cumplimiento de las normas COVID-19, durante una ejecución en tiempo real. De igual forma, el resto de técnicas son capaces de realizar dicha tarea, destacando el buen rendimiento del prototipo con Mediapipe y los malos resultados de \textit{Facial Landmark}, incapaz de detectar el rostro con mascarilla en muchas ocasiones.

De forma análoga, un prototipo de los desarrollados es capaz de reconocer cuando la persona detectada viste mal la mascarilla, siendo el caso de Transfer Learning y Tensorflow. Gracias a la creación de un modelo adaptado a estas necesidades, pero unicamente logrando un buen funcionamiento en situaciones ideales: buena luz y cercanía a la cámara. Finalmente, es importante destacar el último de los objetivos: tratar los falsos positivos. Los prototipos creados a partir de Haar-like feature, Facial Landmarks y Mediapipe presentan dicha característica, ya que realizan detecciones de 'llevar mascarillas' en situaciones donde la persona no la lleva, por ejemplo: cuando la persona obstaculiza la visibilidad de la boca. El motivo proviene de la naturaleza de estos prototipos, dado que basan su funcionamiento en el uso del reconocimiento de características, y al no encontrar la boca se detecta el uso de mascarilla. El único caso apto para este objetivo es el prototipo de Tensorflow, comentado anteriormente.

A lo largo del desarrollo de este trabajo se ha aprendido a emplear herramientas como Python, Tensorflow, frameworks de Python (Mediapipe, Dlib, Numpy) y OpenCV. Además, de profundizar en términos e ideas del \textit{Machine Learning}, visión artificial y \textit{Deep Learning}.

A modo de cierre, se pueden desatacar las siguientes vías futuras para continuar este trabajo:
\begin{itemize}
	\item Realizar más pruebas en distintos dispositivos (ordenadores, móviles, sistemas embebidos).
	\item Ampliar el funcionamiento de los prototipos. Creando modelos más completos.
	\item Utilizar servicios en la nube para implementar los prototipos.
	\item Realizar el estudio de más técnicas.
	\item Añadir más normas COVID-19 en los prototipos.
\end{itemize}