%!TEX root = proyecto.tex

\chapter{Estado del arte}

El reconocimiento facial es un ámbito de la visión artificial, que como su nombre indica, se centra en la búsqueda de rostros humanos dentro de imágenes digitales. Durante años se han desarrollado varias tecnologías capaces de realizar dicha acción, de las que se pueden distinguir dos grupos:

\begin{itemize}
	\item Aplicaciones con sin el uso de \textit{Deep Learning}
	\item Aplicaciones con uso de \textit{Deep Learning}
\end{itemize}

Ambos se centran en las características de los rostros humanos para lograr identificarlos. Esto se conoce como \textit{features}, que corresponden con puntos de la cara muy reconocibles, como: el mentón, ojos, cejas, nariz, etc. Esta técnica fue usada por primera vez en 2001, por Paul Viola y Michael Jones, y desde entonces se ha convertido en unas de las técnicas principales en el reconocimiento facial.

Este ejercicio se complica cuando las imagenes presentan errores naturales en su toma (como baja luz, ruido, etc.) o los individuos visten complementos que tapen sus rasgos faciales. Este es el problema que se va a plantear en este trabajo, buscar una posible solución al reconocimiento facial con complementos faciales, en especifico identificar si las personas llevan mascarillas.

\section{Python y OpenCV}


\section{Aplicaciones sin Deep Learning}


\subsection*{HAAR-like Features \& ADABOOST}


\subsection*{Facial Landmasking}


\section{Aplicaciones con Deep Learning}

¿Que es el deep learning?

\subsection*{YOLO}

\subsection*{TensorFlow}

\subsection*{Keras}

\subsection*{MediaPipe}

\subsection*{IBM Watson}


