%!TEX root = proyecto.tex

\chapter{Introducción}

\lettrine[lines=4]{L}{a} vision artificial  es un ámbito de la informática que surgió hace 60 años, pensado en el estudio del procesamiento digital de las imágenes. De hecho, uno de los primeros acontecimientos que propicio su aparición fue la creación del cable Bartlane, capaz de transmitir una imagen a traves del mar atlantico en los años 1920, con una duración de cerca a una semana.

\section{Historia de la Visión Artificial}

Dentro del entorno del procesamiento de imagenes digitales, se centró la investigacion en recuperar una estructura tridimensional del mundo real a traves de una imagen para conseguir un entendimiento total de la escena que plasma la imagen. A su vez aparecieron varios algoritmos de reconocimiento de lineas, donde uno de ellos fue creado por parte de Huffman en 1971.

Pero el avance de estas investigaciones pronto se ligarían al avance del ordenador. Estos, se podrian dividir en varios acontecimientos importantes, tales como: la invención del transistor por Bell Laboratories en 1948, la invencion de los circuitos integrados en 1958 o la introduccion por parte de IBM de los primeros ordenadores personales en 1981.

Uno de los descubrimientos que inicio este movimiento no fue proveniente de la informática, sino de la psicología. Esta sería una de las principales fuentes sobre el entendimiento de como funciona la vision.  Un par de psicologos, David Hubel y Torsten Wiesel, describieron que el comportamiento de las neuronas encargadas de entender el entorno visual siempre empiezan con estructuras simples como vertices. Más tarde esta idea se convertirá en el principio central del deep learning.

Este estudio se realiza antes de que los ordenadores pudiesen entender imagenes. Russell Kirsch, en 1959, es el primero en desarrollar un aparato que traducia las imagenes en datos que las maquinas pudiesen entender.  Y, Lawrence Roberts en 1963 publica un estudio sobre como las maquinas perciben objetos solidos de tres dimensiones, uno de los avances considerados precursores de la vision artificial moderna.

En 1982, David Marr, le da un punto de vista diferente a la vision artificial. Idea un framework para vision que incluye un esquema con la representacion de las partes principales de la imagen, otro esquema con las superficies y la pronfundidad de la informacion, y un modelo 3D. Al mismo tiempo se desarrollo un red artificial capaz de reconocer patrones, mediante el uso de una red convolucional.

Tras estos descubrimientos se presento un modelo llamado LeNet-5, la primera red convolucional moderna. Este modelo se caracteriza por usar la backpropagación. En los años 1990s la vision artificial cambia totalmente de rumbo y los investigadores pasaron de intentar reconstruir objetos en 3D a intentar detectar objetos mediante sus caracteristicas.

Asimismo, esta epoca de 1980 se centro la investigacion y desarrollo de tecnicas matematicas para el entendimiento de imagenes y las escenas que estan representadas en ellas. Mientras que en 1990 se explotaron todas estas ideas conjuntamente con el desarrollo de la potencia de los ordenadores.

A partir del año 2000 se hacen muchos avances importantes y que actualmente son usados en aplicaciones reales. El primer detector facial llegaría en 2001 creado por Paul Viola y Michael Jones. Ambos consiguieron crear el primero, pero ademas que fuese en tiempo real. Se trata de un clasificador creado a partir de clasificadores mas debiles y busca las caras a partir de dividir la imagen de entrada en rectangulos y realiza un estudio en cascada sobre los clasificadores debiles.

El problema de estos modelos es el uso de informacion para poder entrenarlos, y para esto se creo un proyecto llamado PASCAL VOC. Este creo un dataset estandar para la clasificacion de objetos. Posteriormente, aparecieron más dataset como este. Uno de ellos, en 2010, ImageNet contiene mas de un millon de imagenes para un total de mil objetos. Junto a este dataset aparecio un modelo basado en una red convolucional llamado AlexNet.

\section{Reconocimiento Facial y COVID-19}

En la actualidad, la visión artificial se utiliza en muchos proyectos y esta presente en investigaciones muy importantes para el futuro de la inteligencia artificial en general. Una de ellas se basa en el reconocimiento facial, y es usado en aplicaciones para reconocer personas, aspectos, contador de personas, etc. 

La detección facial se inicia con una imagen arbitraria, con el objetivo de encontrar todas las caras que hay en una imagen, y posteriormente devolver otra con la localización exacta de cada una de las caras. Aunque esta tarea es natural para los humanos, es bastante complicada para los ordenadores. Ya que se encuentran muchos factores que lo dificultan, tales como: la escala, localización, punto de vista, iluminación, lentes, etc.

Existen centenares de investigación/proyectos de detección facial, desde uno de los mas influyentes en los años 2000s, como \textit{Viola and Jones face detection}, a proyectos basados en \textit{Deep Learning}, con tecnologías como \textit{Tensorflow} o \textit{YOLO}.

Pero tras la aparición de COVID-19 y el uso de las mascarillas, la gran mayoría de las aplicaciones o investigaciones que hacían uso de estas tecnologías se han quedado obsoletas o funcionan de una forma mas pobre. Por eso, el objetivo de este trabajo será el estudio de posibilidades que solucionen este problema y terminar consiguiendo un prototipo con el que se puedan detectar rostros humanos que vistan una mascarilla, e incluso poder indicar cuando la viste mal o no la lleven. 


