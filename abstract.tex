%!TEX root = proyecto.tex

\chapter*{Extended abstract}
\addcontentsline{toc}{chapter}{Extended abstract}

The COVID-19 pandemic, also known as the coronavirus pandemic, is an ongoing global pandemic of coronavirus disease 2019 (COVID-19) that affects all the human behaviour. The World Health Organization (WHO) declared a Public Health Emergency of International Concern regarding COVID-19 on 30 January 2020, and later declared a pandemic on 11 March 2020. This is why the WHO plublished an interim guidance to provide a updated guidance on mask use in health care and community settings, and during home care for COVID-19 cases.

This document contains updated evidence and guidance on mask management, virus (COVID-19) transmission, mask use by the public in areas with community and cluster transmission, mask use by the public in areas with community and cluster transmission, mask use during vigorous intensity physical activity, etc. In addition, the World Health Organization (WHO) advises the use of masks as part of a comprehensive package of prevention and control measures to limit the spread of COVID-19. For this reason, this work will propose the use of Computer Vision and Machine Learning techniques for the implementation of prototypes capable of carrying out the control of mask use in real time. The effort will be focused on the study of the following techniques: Haar-like feature, Facial Landmarks, Mediapipe, Transfer Learning / Tensorflow.

The first technique, Haar-like feature, comes  from an investigation directed by Paul Viola and Michael Jones in 2001, focused on real-time facial recognition using these features and a Machine Learning model called Adaboost. Moreover, the paper describes a machine learning approach for visual object detection which is capable of processing images fast and quite accurate. This investigation can be divided into three key concepts. The first is a new image representation called \textit{Integral Image} which allows the features used compute faster during the execution of the detection. The second is a learning algorithm, based on AdaBoost, which selects a small set of features from a larger set features and produces efficient classifiers. The third one is a method for combining those classifiers into a cascade architecture, which allows discard background regions of the input image and spend more computation on promising object-like regions. Therefore, in real-time applications, a detector which use this technique runs at 15 frames per seconds.

[Prototipo implementado]

[Facial landmark]

[Mediapipe]

[Transfer learning]
