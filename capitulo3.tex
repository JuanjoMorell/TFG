%!TEX root = proyecto.tex

\lstset{frame=single,basicstyle=\ttfamily\small}

\chapter{Análisis de objetivos y metodología}

El COVID-19 es un problema que ha provocado en la humanidad incontables problemas, y en el mundo de la visión artificial también, por eso me voy a centrar en la creación de un prototipo que sea capaz de revisar el cumplimiento de las normas COVID impuestas en España y en todo el mundo por la OMS. Concretamente, detectar cuando una persona lleva, de manera correcta, una mascarilla al entrar a un comercio, cine, restaurante, etc. Los objetivos que se plantean para llevar esto a cabo son los siguientes:

\begin{itemize}
	\item Estudio de las tecnologías actuales, para comprobar su comportamiento con uso de mascarilla.
	\item Creación de un prototipo capaz de reconocer rostros y detectar si se lleva mascarilla.
	\item Estudiar la capacidad de que el prototipo pueda identificar si se lleva correctamente la mascarilla. 
	\item Poder detectar la mascarilla, independientemente del tipo que se lleve.
\end{itemize}

\section{Prototipo}

Para cada uno de los apartados del desarrollo de este \textit{TFG} se creará un prototipo con la finalidad de mostrar la tecnología expuesta en el mismo. Con el objetivo final de crea una aplicación que contenga todos los prototipos y se puedan ejecutar de forma sencilla. 

Los prototipos serán probados en varios escenarios de prueba, todos ellos se realizarán en tiempo real. Se contará con una totalidad de tres escenarios:

\begin{itemize}
	\item Detección de mascarillas a una distancia cercana.
	\item Detección de mascarillas a una distancia media.
	\item Detección de mascarillas desde una posición alejada, como la parte superior de una puerta.
\end{itemize}

Asimismo, las pruebas se repetirán en dos dispositivos diferentes. El primero de ellos sin GPU y el segundo con GPU (CUDA). A continuación se muestran las especificaciones de los dispositivos:

\begin{table}[h!]
	\begin{center}
		\begin{tabular}{ |c|c|c|c| } 
			\hline
			 & PC 1 & PC 2 \\
			\hline
			\multirow{3}{4em}{CPU} & Intel i7-1065G7 & Intel i7 \\ 
			& 1.30GHz & 2.60GHz \\ 
			& 8 núcleos & 8 núcleos \\ 
			\hline
			GPU & Intel Iris Plus Graphics  & GTX 980M 2Gb \\
			\hline
			CUDA & NO  & SI \\
			\hline
			RAM & 16 Gb & 8 Gb \\
			\hline
			OS & Ubuntu 18.04 & ? \\
			\hline
		\end{tabular}
		\caption{Entornos de prueba.}
		\label{tab:table1}
	\end{center}
\end{table}

\vspace{0.3cm}

Por último, en los anexos se mostrará una explicación de como se ha implementado más centrado en la programación, conjuntamente con un manual de usuario.