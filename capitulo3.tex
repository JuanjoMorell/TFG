%!TEX root = proyecto.tex

\lstset{frame=single,basicstyle=\ttfamily\small}

\chapter{Análisis de objetivos y metodología}

El COVID-19 es un problema que ha provocado en la humanidad incontables problemas, y en el mundo de la visión artificial también, por eso me voy a centrar en la creación de un prototipo que sea capaz de revisar el cumplimiento de las normas COVID impuestas en España y en todo el mundo por la OMS. Concretamente, detectar cuando una persona lleva, de manera correcta, una mascarilla al entrar a un comercio, cine, restaurante, etc. Los objetivos que se plantean para llevar esto a cabo son los siguientes:

\begin{itemize}
	\item Estudio de las tecnologías actuales, para comprobar su comportamiento con uso de mascarilla.
	\item Creación de un prototipo capaz de reconocer rostros y detectar si se lleva mascarilla.
	\item Estudiar la capacidad de que el prototipo pueda identificar si se lleva correctamente la mascarilla. 
	\item Poder detectar la mascarilla, independientemente del tipo que se lleve.
\end{itemize}

\section{Prototipo}

[Explicar el funcionamiento de como se va a llevar a cabo la creación del prototipo]

\begin{itemize}
	\item Para cada apartado se creará un prototipo especifico para probar dicha tecnología.
	\item Se realizará un estudio de los resultado para cada uno de ellos.
	\item En los anexos se mostrará una explicación de como se ha implementado más centrado en la programación.
\end{itemize}